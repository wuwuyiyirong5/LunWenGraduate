%%---------------------------------------------------------------------------%%
%%------------ 第二章:矩阵根的~Newton~法 -----------------------------------%%
%%---------------------------------------------------------------------------%%


\chapter{计算矩阵~$p$~次根的~Newton~法}
\label{chapter:NM_MatrixRoot}

\newcounter{newlist}
你好
%Newton's method and Halley's method are the ones most widely used
%for computing the principal $p$th root $A^{1/p}$. Their iterative
%forms can be obtained by applying the corresponding classical forms
%to the matrix equation

应用迭代法计算矩阵的~$p$~次根, 实际上是求解如下的计算矩阵方程
\begin{equation}
\label{eq:f(X)=0} f(X) = X^p - A = 0, \quad p \geq 2.
\end{equation}
%Formally, Newton's method can be written as follows:
若假设初始迭代点~$X_0$~与矩阵~$A$~是可交换的, 且对于任意的~$k \geq
0$, $X_k$~都是非奇异的, 则计算上述矩阵方程的~Newton~法可写成:
\begin{equation}
\label{it:NM} X_{k+1}= N(X_k),\ \ \ k = 0,1,2,\ldots,
\end{equation}
%provided $X_0$ commutes with $A$ and that $X_k$ is nonsingular for
%any $k \geq 0$, where
其中
\begin{equation}
\label{it:NM_fun} N(X) := \frac{1}{p}X\left[(p-1)\I +
AX^{-p}\right].
\end{equation}


本章将研究通过~Newton~法来计算矩阵~$p$~次根的一个新的 收敛性结果。
新的结果比其他现有的结果在计算矩阵~$p$~次根时有更好的计算效果。
本章中总假定整数~$p \geq 2$.









%In this paper, we study refined convergence results on Newton's
%method and Halley's method for computing the principal $p$th root of
%a matrix. The new convergence regions which contain all the
%eigenvalues of $A$ guarantee (\ref{ineq:def_conv_order}) holds for
%all $k \geq 0$ for both methods. In Section 2 we will state our main
%results for both methods. And then the result concerning Newton's
%method will be proved in Section 3, while the one for Halley's
%method can be proved by using the similar approaches.





\section{收敛性结果}


在给出本章的主要收敛结果前, 先引入一些记号。令
\begin{equation}
\label{fun:phi(z)_NM} \phi_1(z) := 1 - u^p(z)(1 - z), \quad z \in
\MCD_{0,1},
\end{equation}
其中
\begin{equation*} u(z) := \frac{1}{1 - \frac{1}{p}z}, \quad z
\in \MCD_{0,1},
\end{equation*}
及
\begin{equation}
\label{set:D0_NM} \MCD_{0,1} := \left\{z \in \CS: |z| < p \right\}.
\end{equation}
定义
\begin{equation}
\label{set:R_NM} \MCR_1 := \left\{z\in \CS: 1 - z \in
\overline{\MCD}_1 \bigcup \left(\MCD_{0,1} \bigcap
\MCD_{2,1}\right)\right\}
\end{equation}
及
\begin{equation}
\label{set:R_hat_NM} \widehat{\MCR}_1 := \left\{z\in\CS: 1-z \in
\mathcal {D}_1 \bigcup \left(\mathcal {D}_{0,1} \bigcap \mathcal
{D}_{2,1}\right)\right\},
\end{equation}
其中~$\MCD_{0,1}$~由~(\ref{set:D0_NM})~给出, $\MCD_1$~定义为
\begin{equation}
\label{set:D1_NM} \MCD_1 := \{z \in \CS: |z| < 1\},
\end{equation}
$\overline{\MCD}_1$~为~$\MCD_1$~的闭包, 而
\begin{equation}
\label{set:D2_NM} \MCD_{2,1} := \left\{z \in \CS: \sup_{m \geq 2}
\left\{\frac{|S_{1,m}(z)|}{|z|}\right\} \cdot \frac{|z|
+|\phi_1(z)|}{\big||z| - |\phi_1(z)|\big|} < 1 \right\},
\end{equation}
其中~$\phi_\nu$~ 由~(\ref{fun:phi(z)_NM})~给出,
\begin{equation}
\label{ser:Sm(z)_NM} S_{1,m}(z) = \sum\limits_{j = 2}^{m} c_{1,j}
z^{j}, \quad z \in \MCD_{0,1},
\end{equation}
其中
\begin{equation*}
\label{cons:NM_c1j} c_{1,j} =
\frac{p(p+1)\cdot(p+j-2)}{(j-1)!p^{j-1}} -
\frac{p(p+1)\cdot(p+j-1)}{j!p^j} > 0, \quad j=2,3,\ldots,
\end{equation*}
并满足
$$
\sum_{j=2}^\infty c_{1,j}=1,
$$
故 ~$\phi_1(z)$ 有如下的~Maclaurin~级数形式
\begin{equation}
\label{fun:phi1(z)_series} \phi_1(z) = \sum_{j=2}^\infty c_{1,j}
z^j, \quad z \in \MCD_{0,1},
\end{equation}
其中 ~$\MCD_{0,1}$~ 由 ~(\ref{set:D0_NM})~给出.



下面给出应用~Newton~法~(\ref{it:NM})~来计算矩阵主~$p$~次根时的收敛性结果:


\begin{theorem}
\label{th:MatrixNMCon} 设~$\MCR_1$~由~$(\ref{set:R_NM})$~定义,
而~$\widehat{\MCR}_1$~ 由~$(\ref{set:R_hat_NM})$~定义。 如果矩阵~$A
\in \CS^{n \times n}$~的所有特征值都属于~$\MCR_1$~
且所有零特征值都是半单的, 那么以~$X_0 =
\I$~为初始点的~Newton~法~$(\ref{it:NM})$~迭代所产生的矩阵序列~$\{X_k\}$~收敛于
矩阵~$A$~的主~$p$~次根~$A^{1/p}$。特别地,
如果~$A$~的所有特征值都属于~$\widehat{\MCR}_1\backslash\{0\}$,
那么~Newton~法~$(\ref{it:NM})$~是二次收敛的。
\end{theorem}

\begin{remark}
图~\ref{fig:NM_ConvReg}~所示的是,
由~(\ref{set:R_NM})~所定义的~$\MCR_1$~ 的近似收敛域的九种情形,
其中红色表示的是收敛域~$\{z \in \CS: 1-z \in \overline{\MCD}_1\}$,
而蓝色表示的是收敛域~$\{z\in \CS: 1-z \in
\MCD_{0,1}\bigcap\MCD_{2,1}\}$。由该图可知, 对于某一固定的~$p$,
当~$m$~分别取 ~$m=20, 100, 500$~时,
所得到的近似收敛域~(见~(a)-(c),或 ~(d)-(f), 或
~(g)-(i))~几乎一样。为此, 在实际计算中只需要取~$m=20$~即可。
\end{remark}

\begin{figure}[h!]
\centering
\subfigure[]{\includegraphics[width=0.3\textwidth]{fig_NM_ConvReg_p25m20.eps}}
\subfigure[]{\includegraphics[width=0.3\textwidth]{fig_NM_ConvReg_p25m100.eps}}
\subfigure[]{\includegraphics[width=0.3\textwidth]{fig_NM_ConvReg_p25m500.eps}}\\
\subfigure[]{\includegraphics[width=0.3\textwidth]{fig_NM_ConvReg_p100m20.eps}}
\subfigure[]{\includegraphics[width=0.3\textwidth]{fig_NM_ConvReg_p100m100.eps}}
\subfigure[]{\includegraphics[width=0.3\textwidth]{fig_NM_ConvReg_p100m500.eps}}\\
\subfigure[]{\includegraphics[width=0.3\textwidth]{fig_NM_ConvReg_p400m20.eps}}
\subfigure[]{\includegraphics[width=0.3\textwidth]{fig_NM_ConvReg_p400m100.eps}}
\subfigure[]{\includegraphics[width=0.3\textwidth]{fig_NM_ConvReg_p400m500.eps}}\\
%\caption{The approximating regions of $\MCR_1$ defined in
%(\ref{set:R_NM}) for $p = 25, 100, 400$ and $m = 20, 100, 500$,
%where the red and blue regions denote the sets $\{z\in\CS: 1-z \in
%\overline{\MCD}_1\}$ and $\{z\in \CS: 1-z \in \MCD_{0,1} \bigcap
%\MCD_{2,1}\}$, respectively. }
\caption{由~(\ref{set:R_NM})~给出的~$\MCR_1$~在九种不同情形~($p$~分别取~25,
100, 400~及~$m$~分别取~20, 100, 500)~下的近似域,
其中红色域表示集合~$\{z\in\CS: 1-z \in \overline{\MCD}_1\}$},
而蓝色域表示集合~$\{z\in \CS: 1-z \in \MCD_{0,1}
\bigcap\MCD_{2,1}\}.$\label{fig:NM_ConvReg}
\end{figure}















%\section{定理~\ref{th:MatrixNMCon}~的证明}


%In this section, we will prove in detail Theorem
%\ref{th:MatrixNMCon}, the results on convergence and convergence
%order for Newton's method (\ref{it:NM}). Theorem
%\ref{th:MatrixHMCon} for Halley's method (\ref{it:HM}) can be
%similarly proved by using the analysis approaches. We omit the
%details.



\section{预备引理}

在证明定理~\ref{th:MatrixNMCon}~前, 需要一些有用的预备引理.

给定任意一个复数~$\lambda \in \CS$, 令

%For a complex number $\lambda \in \CS$, let
\begin{equation}
\label{fun:f(z)} f(z) := z^p - \lambda, \ \ \ z \in \CS
\end{equation}
及
\begin{equation}
\label{fun:r(z)} r(z,\lambda) := 1 - \lambda z^{-p}, \ \ \ z \in
\CS\backslash\{0\}.
\end{equation}

\begin{lemma}
\label{lem:NM_r(N(z))_r(z)} 设~$r(z,\lambda)$~由~$(\ref{fun:r(z)})$~
定义. 对于某个复数~$z \in \CS\backslash\{0\}$, 若~$r(z,\lambda) \in
\mathcal {D}_{0,1}\backslash\{1\}$,
其中~$\MCD_{0,1}$~由~$(\ref{set:D0_NM})$~给出,
则由~$(\ref{it:NM_fun})$~的标量形式所得到的~$N(z)$~有定义且
\begin{equation}
\label{eq:r(N(z))_1} r(N(z),\lambda) = \phi_1(r(z,\lambda)),
\end{equation}
其中~$\phi_1(z)$~由~$(\ref{fun:phi(z)_NM})$~定义. 此外,
有如下的估计: 当~$r(z,\lambda) \in
\overline{\MCD}_1\backslash\{0,1\}$~时,
\begin{equation}
\label{ineq:abs_r(N(z))_1} |r(N(z),\lambda)| < |r(z,\lambda)|^2.
\end{equation}
当~$r(z,\lambda) \in \mathcal {D}_{0,1}\backslash\{1\}$~时,
\begin{equation}
\label{ineq:abs_r(N(z))_2} |r(N(z),\lambda)| \leq \sup_{m \geq
2}\left\{\left|\frac{S_{1,m}(u)}{u^2}\right|\right\}\cdot \frac{|u|
+ |r(z,\lambda)|}{|u| - |r(z,\lambda)|} \cdot |r(z,\lambda)|^2,
\end{equation}
其中~$\overline{\MCD}_1$~为~$\MCD_1$~的闭包,
而~$\MCD_1$~由~$(\ref{set:D1_NM})$~定义,
$S_{1,m}(u)$~由~$(\ref{ser:Sm(z)_NM})$~定义且对于每一个~$u \in
\MCD_{0,1}$~均满足~$|u| > |r(z,\lambda)|$, $m \geq 2$.
\end{lemma}

%\begin{lemma}
%\label{lem:NM_r(N(z))_r(z)} Let $r(z,\lambda)$ be defined in
%$(\ref{fun:r(z)})$. If $r(z,\lambda) \in \mathcal
%{D}_{0,1}\backslash\{1\}$ for some $z \in \CS\backslash\{0\}$, where
%$\MCD_{0,1}$ is defined in $(\ref{set:D0_NM})$, then $N(z)$
%generated by the scalar case of $(\ref{it:NM_fun})$ for $f(z)$
%defined in $(\ref{fun:f(z)})$ exists and
%\begin{equation}
%\label{eq:r(N(z))_1} r(N(z),\lambda) = \phi_1(r(z,\lambda)),
%\end{equation}
%where $\phi_1(z)$ is defined by $(\ref{fun:phi(z)_NM})$. Moreover,
%we have
%\begin{numcases}{}
%|r(N(z),\lambda)| < |r(z,\lambda)|^2, &
%if \ $r(z,\lambda) \in \overline{\MCD}_1\backslash\{0,1\}$,\ \ \ \label{ineq:abs_r(N(z))_1}\\
%|r(N(z),\lambda)| \leq \sup_{m \geq
%2}\left\{\left|\frac{S_{1,m}(u)}{u^2}\right|\right\}\cdot \frac{|u|
%+ |r(z,\lambda)|}{|u| - |r(z,\lambda)|} \cdot |r(z,\lambda)|^2, &
%if \ $r(z,\lambda) \in \mathcal {D}_{0,1}\backslash\{1\}$,
%\label{ineq:abs_r(N(z))_2}
%\end{numcases}
%where $\overline{\MCD}_1$ is the closure of $\MCD_1$ defined in
%$(\ref{set:D1_NM})$, and $S_{1,m}(u)$ is given in
%$(\ref{ser:Sm(z)_NM})$ for each $u \in \MCD_{0,1}$ satisfying $|u| >
%|r(z,\lambda)|$, $m \geq 2$.
%\end{lemma}


\begin{proof}

%For any $z \in \CS\backslash\{0\}$, $N(z)$ exists by
%(\ref{it:NM_fun}). Note that

对于任意的复数~$z \in \CS\backslash\{0\}$,
由~(\ref{it:NM_fun})~知~$N(z)$~是存在的. 由于

\begin{equation}
\label{eq:N(z)} N(z) = \frac{1}{p}z\left[(p-1)+\lambda z^{-p}\right]
= z\left[1+\frac{1}{p}r(z,\lambda)\right] \neq 0.
\end{equation}
%It follows that
故
\begin{equation}
\label{eq:r(N(z))} r(N(z),\lambda) =
1-\left[1+\frac{1}{p}r(z,\lambda)\right]^{-p}(1-r(z,\lambda)) =
\phi_1(r(z,\lambda)),
\end{equation}
%which shows (\ref{eq:r(N(z))_1}) holds.
即得~(\ref{eq:r(N(z))_1})~是成立的.

%If $r(z,\lambda) \in \overline{\MCD}_1\backslash\{0,1\} \subset
%\MCD_{0,1}$, then, by (\ref{fun:phi1(z)_series}) and
%(\ref{eq:r(N(z))}), one has that

若~$r(z,\lambda) \in \overline{\MCD}_1\backslash\{0,1\} \subset
\MCD_{0,1}$, 则由~(\ref{fun:phi1(z)_series})~及~(\ref{eq:r(N(z))}),
并注意到对于任意的~$w \in \overline{\MCD}_1\backslash\{1\}$,
不等式~$|c_3 + c_4 w| < c_3 + c_4$~都是成立的, 于是可得
\begin{align}
|r(N(z),\lambda)| & = |\phi(r(z,\lambda))| \nonumber\\
& = |r(z,\lambda)|^2 \left| \sum_{j =
2}^\infty c_{1,j} r^{j-2}(z,\lambda) \right| \nonumber\\
& \leq |r(z,\lambda)|^2 \left[|c_{1,3} + c_{1,4} r(z,\lambda)| +
\sum_{j = 4}^\infty c_{1,j} |r(z,\lambda)|^{j-2}
\right] \nonumber\\
& < |r(z,\lambda)|^2 \sum_{j = 2}^\infty c_{1,j} =
|r(z,\lambda)|^2\nonumber\\
& \leq |r(z,\lambda)|. \label{ineq:abs_r(N(z))}
\end{align}
%from the fact that $|c_3 + c_4 w| < c_3 + c_4$ for all $w \in
%\overline{\MCD}_1\backslash\{1\}$. Thus, (\ref{ineq:abs_r(N(z))_1})
%is proved.
即~(\ref{ineq:abs_r(N(z))_1})~是成立的.


%If $r(z,\lambda) \in \mathcal {D}_0\backslash\{1\}$, based on
%(\ref{fun:phi1(z)_series}) and (\ref{eq:r(N(z))}) again, we have

若~$r(z,\lambda) \in \mathcal {D}_0\backslash\{1\}$,
则由~(\ref{fun:phi1(z)_series})~及~(\ref{eq:r(N(z))})~可知,
对于任意的 ~$u \in \CS\backslash\{0\}$~有
\begin{align*}
r(N(z),\lambda) & = \phi_1(r(z,\lambda))\\
& = \sum_{j=2}^\infty c_{1,j} r^j(z,\lambda)\\
& = \left[\sum_{j=2}^\infty c_{1,j} u^{j-2}
\left(\frac{r(z,\lambda)}{u}\right)^{j-2}\right] \cdot
r^2(z,\lambda).
\end{align*}
%holds for any $u \in \CS\backslash\{0\}$. Since, for any $m \geq 2$,
对于任意的~$m \geq 2$, 应用~Abel~变换可得
\begin{align*}
\sum_{j=2}^m c_{1,j} u^{j-2}
\left(\frac{r(z,\lambda)}{u}\right)^{j-2} & = \sum_{j=2}^{m-1}
\left(\sum_{\ell=2}^j c_{1,\ell} u^{\ell-2}\right)\left(1 -
\frac{r(z,\lambda)}{u}\right)\left(\frac{r(z,\lambda)}{u}\right)^{j-2}\\
& \ \ \ + \left(\sum_{\ell = 2}^m c_{1,\ell} u^{\ell-2}\right)\cdot
\left(\frac{r(z,\lambda)}{u}\right)^{m-2}.
\end{align*}
%by Abel transformation, we have
于是
\begin{align*}
\left|\sum_{j=2}^m c_{1,j} u^{j-2}
\left(\frac{r(z,\lambda)}{u}\right)^{j-2}\right| & \leq \sup_{2\leq
j \leq m-1} \left\{\left|\sum_{\ell=2}^j c_{1,\ell}
u^{\ell-2}\right|\right\} \cdot \left|1 -
\frac{r(z,\lambda)}{u}\right| \cdot \sum_{j=2}^{m-1}
\left|\frac{r(z,\lambda)}{u}\right|^{j-2}\\
& \ \ \ + \left|\frac{S_{1,m}(u)}{u^2} \right|\cdot
\left|\frac{r(z,\lambda)}{u}\right|^{m-2}, \quad m >2.
\end{align*}
%Then, letting $m \to \infty$ in the above inequality, it follows
%that
在上述不等式中令~$m \to \infty$, 则对于任意的~$r(z,\lambda) \in
\mathcal {D}_0\backslash\{1\}$~及满足关系~$|u|
> |r(z,\lambda)|$~的任意复数~$u
\in \mathcal {D}_0$, 有
\begin{align*}
|r(E(z),\lambda)| & \leq \sup_{m \geq 2} \left\{\left|\sum_{j=2}^m
c_{1,j} u^{j-2}\right|\right\} \cdot \left|1 -
\frac{r(z,\lambda)}{u}\right| \cdot
\sum_{m=2}^\infty \left|\frac{r(z,\lambda)}{u}\right|^{m-2}\cdot |r(z,\lambda)|^2 \\
& = \sup_{m \geq 2}
\left\{\left|\frac{S_{1,m}(u)}{u^2}\right|\right\} \cdot
\frac{\left|1 - \frac{r(z,\lambda)}{u}\right|}{1 -
\left|\frac{r(z,\lambda)}{u}\right|} \cdot |r(z,\lambda)|^2 \\
& = \sup_{m \geq 2}
\left\{\left|\frac{S_{1,m}(u)}{u^2}\right|\right\} \cdot \frac{|u| +
|r(z,\lambda)|}{|u| - \left|r(z,\lambda)\right|} \cdot
|r(z,\lambda)|^2.
\end{align*}
%for any $r(z,\lambda) \in \mathcal {D}_0\backslash\{1\}$ and any $u
%\in \mathcal {D}_0$ subject to $|u|
%> |r(z,\lambda)|$, which verifies (\ref{ineq:abs_r(N(z))_2}). The
%proof is completed.
从而~(\ref{ineq:abs_r(N(z))_2})~得证. 证完.
\end{proof}



\begin{lemma}
\label{lem:NM_r(z)_convergence1} %Let $r(z,\lambda)$ be defined in
%$(\ref{fun:r(z)})$. If $r(z_0,\lambda) \in
%\overline{\MCD}_1\backslash\{0,1\}$ for some $z_0 \in
%\CS\backslash\{0\}$, then the sequence $\{z_k\}$ starting from $z_0$
%generated by the scalar case of $(\ref{it:NM})$ for solving
%$(\ref{fun:f(z)})$ exists,

设~$r(z,\lambda)$~由~$(\ref{fun:r(z)})$~定义. 对于某个复数~$z_0 \in
\CS\backslash\{0\}$, 若~$r(z_0,\lambda) \in
\overline{\MCD}_1\backslash\{0,1\}$,
则由~$(\ref{it:NM})$~的标量形式~$($以~$z_0$~为初始点$)$~迭代产生的序列~$\{z_k\}$~
是有定义的, 且有估计式:
\begin{equation}
\label{ineq:NM_abs_r(zk)_1} |r(z_k,\lambda)| \leq
q_1^{2^{k-1}}(z_0), \ \ \ k = 1, 2, \ldots,
\end{equation}
%where
其中
\begin{equation}
\label{cons:NM_q1(z0)} q_1(z_0) = q_1(z_0,\lambda) :=
\left|\sum_{j=2}^\infty c_{1,j} r^{j-1}(z_0,\lambda)\right| < 1.
\end{equation}
%and so $|r(z_k,\lambda)| \to 0$ with order $2$ as $k \to \infty$.
由此可知, 当~$k \to \infty$~时~$|r(z_k,\lambda)|$~收敛于~0,
且收敛速度是二阶的.
\end{lemma}

\begin{proof}
%For $z_0$ chosen, $q_1(z_0) < 1$ follows from the same arguments
%used in (\ref{ineq:abs_r(N(z))}). By (\ref{ineq:abs_r(N(z))_1}) in
%Lemma \ref{lem:NM_r(N(z))_r(z)}, $z_1 = E(z_0)$ exists and

对于给定的~$z_0$,
类似于~(\ref{ineq:abs_r(N(z))})~的处理方式可知~$q_1(z_0) < 1$.
根据引理~\ref{lem:NM_r(N(z))_r(z)}~中的~(\ref{ineq:abs_r(N(z))_1})~知~$z_1
= E(z_0)$~是存在的并且有
$$
|r(z_1,\lambda)| = q_1(z_0)|r(z_0,\lambda)| \leq q_1(z_0) < 1.
$$
%Suppose that $z_k$ exists and (\ref{ineq:NM_abs_r(zk)_1}) holds for
%some $k \geq 1$, then by Lemma \ref{lem:NM_r(N(z))_r(z)} again,
%$z_{k+1} = E(z_k)$ exists and
对于某一~$k \geq 1$,
假设~$z_k$~是存在的且~(\ref{ineq:NM_abs_r(zk)_1})~是成立的,
则由引理~\ref{lem:NM_r(N(z))_r(z)}~可知~$z_{k+1} =
E(z_k)$~是存在的并且有
\begin{equation*}
|r(z_{k+1},\lambda)| < |r(z_{k},\lambda)|^2 \leq
\left[q_1^{2^{k-1}}(z_0)\right]^2 = q_1^{2^{k}}(z_0).
\end{equation*}
%Thus, (\ref{ineq:NM_abs_r(zk)_1}) holds for $k+1$. By induction,
%$\{z_k\}$ exists and (\ref{ineq:NM_abs_r(zk)_1}) holds. Furthermore,
%$r(z_k,\lambda) \to 0$ with order 2 as $k \to \infty$, which
%completes the proof.
由此知, (\ref{ineq:NM_abs_r(zk)_1})~对于~$k+1$~的情形仍然成立.
于是由归纳法知,
序列~$\{z_k\}$~是存在的且~(\ref{ineq:NM_abs_r(zk)_1})~恒成立. 证完.
\end{proof}



%The following lemma says that, besides in $
%\overline{\MCD}_1\backslash\{0,1\}$, it also guarantees
%$r(z_k,\lambda)$ converges quadratically to 0 as $k \to \infty$ when
%$r(z_0,\lambda) \in \MCD_{2,1} \bigcap \MCD_{0,1}\backslash\{1\}$
%for some $z_0 \in \CS\backslash\{0\}$.

对于某个复数~$z_0 \in \CS\backslash\{0\}$,
引理~\ref{lem:NM_r(z)_convergence1}~ 说明当~$r(z_0,\lambda) \in
\overline{\MCD}_1\backslash\{0,1\}$~时可保证~$r(z_k,\lambda)$~是二阶收敛于~0,
下面的引理表明, 除此之外, 当~$r(z_0,\lambda) \in \MCD_{2,1} \bigcap
\MCD_{0,1}\backslash\{1\}$~时,
仍然可以保证~$r(z_k,\lambda)$~是二阶收敛于~0.

\begin{lemma}
\label{lem:NM_r(z)_convergence2} %For any $z_0 \in
%\CS\backslash\{0\}$ satisfying $r(z_0,\lambda) \in \MCD_{2,1}
%\bigcap \MCD_{0,1}\backslash\{1\}$ and

设~$r(z,\lambda)$~由~$(\ref{fun:r(z)})$~定义. 对于某一复数~$z_0 \in
\CS\backslash\{0\}$, 若~$r(z_0,\lambda) \in \MCD_{2,1} \bigcap
\MCD_{0,1}\backslash\{1\}$~且
\begin{equation}
\label{cons:NM_q2(z0)} q_2(z_0) = q_2(z_0,\lambda) := \sup_{m \geq
2} \left\{\frac{|S_{1,m}(r(z_0,\lambda))|}{|r(z_0,\lambda)|}\right\}
\cdot \frac{|r(z_0,\lambda)|
+|\phi_1(r(z_0,\lambda))|}{|r(z_0,\lambda)| -
|\phi_1(r(z_0,\lambda))|} < 1,
\end{equation}
%where $\MCD_{2,1}$ is defined in $(\ref{set:D2_NM})$, the sequence
%$\{z_k\}$ generated by the scalar form of $(\ref{it:NM})$ starting
%from $z_0$ for solving $(\ref{fun:f(z)})$ exists,
其中~$\MCD_{2,1}$~由~$(\ref{set:D2_NM})$~定义,
则由~$(\ref{it:NM})$~的标量形式~$($以~$z_0$~为初始点$)$~
迭代产生的序列~$\{z_k\}$~是有意义的,
且有如下估计:
\begin{equation}
\label{ineq:NM_abs_r(zk)_2} |r(z_k,\lambda)| \leq q_2^{2^k-1}(z_0)
\cdot |r(z_0,\lambda)|, \quad k = 0, 1, \ldots.
\end{equation}
%and so $|r(z_k,\lambda)| \to 0$ with order $2$ as $k \to \infty$.
由此可知, 当~$k \to \infty$~时~$|r(z_k,\lambda)|$~收敛于~$0$,
且收敛速度是二阶的.
\end{lemma}

\begin{proof}
%For $z_0$ chosen, we have $r(z_0,\lambda) \in \mathcal
%{D}_{0,1}\backslash\{1\}$. So, $z_1 = N(z_0)$ exists and $z_1 \neq
%0$ by (\ref{eq:N(z)}). Recall that $S_{1,m}(r(z_0,\lambda)) \to
%\phi_1(r(z_0,\lambda))$ as $k\to\infty$ implies
显然, 对于给定的~$z_0$~有~$r(z_0,\lambda) \in \mathcal
{D}_{0,1}\backslash\{1\}$. 故~$z_1 =
N(z_0)$~存在且由~(\ref{eq:N(z)})~知~$z_1 \neq 0$. 注意到,
当~$k\to\infty$~时~$S_{1,m}(r(z_0,\lambda)) \to
\phi_1(r(z_0,\lambda))$, 故由~(\ref{cons:NM_q2(z0)})~可得
$$
\left|\frac{\phi_1(r(z_0,\lambda))}{r(z_0,\lambda)}\right| \leq
\sup_{m\geq2}
\left\{\left|\frac{S_{1,m}(r(z_0,\lambda))}{r(z_0,\lambda)}\right|\right\}
< q_2^2(z_0)<1,
$$
%by (\ref{cons:NM_q2(z0)}), we have
于是有
$$
|r(z_1,\lambda)| = |\phi_1(r(z_0,\lambda))| =
\left|\frac{\phi_1(r(z_0,\lambda))}{r(z_0,\lambda)}r(z_0,\lambda)\right|
< q_2^2(z_0)\cdot|r(z_0,\lambda)|,
$$
%and (\ref{ineq:NM_abs_r(zk)_2}) holds for $k=1$. Assume $z_0, z_1,
%\ldots, z_k$ exist and satisfy (\ref{ineq:NM_abs_r(zk)_2}). Then
即当~$k=1$~时~(\ref{ineq:NM_abs_r(zk)_2})~是成立的. 现假设~$z_0,
z_1, \ldots, z_k$~都是存在的且满足~(\ref{ineq:NM_abs_r(zk)_2}), 则
$$
|r(z_k,\lambda)| \leq q_2^2(z_0)|r(z_0,\lambda)|< |r(z_0,\lambda)| <
p.
$$
%So, by Lemma \ref{lem:NM_r(N(z))_r(z)} with $u = r(z_0,\lambda)$,
%$z_{k+1} = N(z_k)$ exists, $z_{k+1} \neq 0$ and
由引理~\ref{lem:NM_r(N(z))_r(z)}~ (取~$u =
r(z_0,\lambda)$)~可知~$z_{k+1} = N(z_k)$~是存在的且~$z_{k+1} \neq
0$. 进一步有
\begin{align*}
|r(z_{k+1},\lambda)| & = |\phi(r(z_k,\lambda))| \\
& \leq \sup_{m \geq 2}
\left\{\frac{|S_{1,m}(r(z_0,\lambda))|}{|r(z_0,\lambda)|^2}\right\}\cdot
\frac{|r(z_0,\lambda)|
+|\phi_1(r(z_0,\lambda))|}{\Big||r(z_0,\lambda)| -
|\phi(r(z_0,\lambda))|\Big|} \cdot |r(z_k,\lambda)|^2 \\
& \leq \sup_{m \geq 2}
\left\{\frac{|S_{1,m}(r(z_0,\lambda))|}{|r(z_0,\lambda)|^2}\right\}\cdot
\frac{|r(z_0,\lambda)|
+|\phi_1(r(z_0,\lambda))|}{\Big||r(z_0,\lambda)| -
|\phi_1(r(z_0,\lambda))|\Big|}
\left[q_2^{2^k-1}(z_0)\right]^2\cdot|r(z_0,\lambda)|^2 \\
& = \sup_{m \geq 2}
\left\{\frac{|S_{1,m}(r(z_0,\lambda))|}{|r(z_0,\lambda)|}\right\}\cdot
\frac{|r(z_0,\lambda)|
+|\phi_1(r(z_0,\lambda))|}{\Big||r(z_0,\lambda)| -
|\phi_1(r(z_0,\lambda))|\Big|} \cdot [q_2(z_0)]^{2^{k+1}-2} \cdot
|r(z_0,\lambda)| \\
& = [q_2(z_0)]^{2^{k+1}-1} \cdot |r(z_0)|,
\end{align*}
%which shows (\ref{ineq:NM_abs_r(zk)_2}) by induction. The proof is
%completed.
由此, 根据归纳法知~(\ref{ineq:NM_abs_r(zk)_2})~得证. 证完.
\end{proof}



%Now, based on the above lemmas, we can obtain the following
%convergence results for scalar Newton's method (\ref{it:NM}). Define

基于上述几个引理,
可以得到如下的关于~(标量形式的)~Newton~法~(\ref{it:NM})~的收敛性结果.
为此, 定义
\begin{equation}
\label{set:R11} \MCR_{1,1} := \big\{\lambda \in \CS: r(z_0,\lambda)
\in \overline{\MCD}_1 \text{ for some } z_0 \in
\CS\backslash\{0\}\big\},
\end{equation}
及
\begin{equation}
\label{set:R12} \MCR_{1,2} := \left\{\lambda \in \CS: r(z_0,\lambda)
\in \MCD_{0,1} \bigcap \MCD_{2,1} \text{ for some } z_0 \in
\CS\backslash\{0\}\right\},
\end{equation}
%where $\overline{\MCD}_1$ is the closure of $\MCD_1$ defined in
%(\ref{set:D1_NM}), $\mathcal {D}_{0,1}$ and $\mathcal {D}_{2,1}$ are
%defined in (\ref{set:D0_NM}) and (\ref{set:D2_NM}), respectively.
其中~$\overline{\MCD}_1$~为~$\MCD_1$~的闭包,
而~$\MCD_1$~由~(\ref{set:D1_NM})~定义, $\mathcal
{D}_{0,1}$~和~$\mathcal {D}_{2,1}$~分别由~(\ref{set:D0_NM})~ 和
~(\ref{set:D2_NM})~定义.


\begin{lemma}
\label{lem:ScalarNMCon1} %For any $\lambda \in \MCR_{1,1} \bigcup
%\MCR_{1,2}$, where $\MCR_{1,1}$ and $\MCR_{1,2}$ are defined by
%$(\ref{set:R11})$ and $(\ref{set:R12})$, respectively, the sequence
%$\{z_k(\lambda)\}$ generated by scalar Newton iteration
%$(\ref{it:NM})$ with some $z_0 \in \CS\backslash\{0\}$ for solving
%$(\ref{fun:f(z)})$ converges to the principal $p$th root
%$\lambda^{1/p}$. Moreover, if $\lambda \neq 0$, then the convergence
%order is $2$.
对于任意的~$\lambda \in \MCR_{1,1} \bigcup \MCR_{1,2}$,
其中~$\MCR_{1,1}$~和~$\MCR_{1,2}$~分别由
~$(\ref{set:R11})$~和~$(\ref{set:R12})$~定义, 以~$z_0 \in
\CS\backslash\{0\}$~为初始点的标量~Newton~法~$(\ref{it:NM})$~迭代所产生的序列
~$\{z_k(\lambda)\}$~收敛于~$\lambda$~的主~$p$~次根~$\lambda^{1/p}$.
此外, 若~$\lambda \neq 0$, 则收敛速度是二阶的.
\end{lemma}


\begin{proof}
%We prove this lemma by four steps as follows.

下面分四步来证明该引理.

%Step 1. Suppose $\MCR_c$ is any closed domain in $\MCR_{1,1}$ or
%$\MCR_{1,2}$ and that $0 \not\in \MCR_c$. We will prove in this step
%$\{z_k(\lambda)\}$ converges uniformly to $z(\lambda)$, a $p$th root
%of each $\lambda \in \MCR_c$, and that $z(\lambda)$ exists for each
%$\lambda \in \MCR_{1,1}\bigcup\MCR_{1,2}$.

第一步.
假设~$\MCR_c$~为属于~$\MCR_{1,1}$~或~$\MCR_{1,2}$~的任一闭区域且~$0
\not\in \MCR_c$. 首先证明, 对于任意的~$\lambda \in \MCR_c$,
序列~$\{z_k(\lambda)\}$~一致收敛于~$\lambda$~的一个~$p$~次根~$z(\lambda)$.


%We write $r(z_k,\lambda) \triangleq r(z_k(\lambda),\lambda)$ for
%short below. Since $ \sum\limits_{j=2}^\infty c_{1,j}
%r^j(z_0,\lambda)$ is analytic for each $\lambda \in
%\MCR_{1,1}\bigcup\MCR_{1,2}$ and $\MCR_c \subset
%\MCR_{1,1}\bigcup\MCR_{1,2}$ is bounded, there is a
%$\widehat{\lambda} \in
%\partial\MCR_c$ such that

记
$$
r(z_k,\lambda) \triangleq r(z_k(\lambda),\lambda).
$$
对于任意的~$\lambda \in \MCR_{1,1}\bigcup\MCR_{1,2}$, 由于级数
$$
\sum\limits_{j=2}^\infty c_{1,j} r^j(z_0,\lambda)
$$
是解析的, 且~$\MCR_c \subset \MCR_{1,1}\bigcup\MCR_{1,2}$~是有界的,
故根据解析函数的最大模定理知, 存在~$\widehat{\lambda} \in
\partial\MCR_c$~使得
\begin{equation}
\label{eq:max_mod_1} \left|\sum_{j=2}^\infty c_{1,j}
r^j(z_0,\widehat{\lambda})\right| = \max_{\lambda \in
\MCR_c}\left|\sum_{j=2}^\infty c_{1,j} r^j(z_0,\lambda)\right|.
\end{equation}
%by the theorem of maximum modulus of analytic function. Let
令
$$
q(z_0) := \left\{
\begin{array}{ll}
\displaystyle \max_{\lambda \in \MCR_c}q_1(z_0,\lambda), & \mbox{if
$\MCR_c
\subset \MCR_{1,1}$,} \\
\displaystyle \max_{\lambda \in \MCR_c}q_2(z_0,\lambda), & \mbox{if
$\MCR_c \subset \MCR_{1,2}$},
\end{array}
\right.
$$
%where $q_1(z_0,\lambda)$ and $q_2(z_0,\lambda)$ are defined in
%(\ref{cons:NM_q1(z0)}) and (\ref{cons:NM_q2(z0)}), respectively.
%Then
其中~$q_1(z_0,\lambda)$~和~$q_2(z_0,\lambda)$~分别由~(\ref{cons:NM_q1(z0)})~
和~(\ref{cons:NM_q2(z0)})~定义. 则由~(\ref{eq:max_mod_1}),
引理~\ref{lem:NM_r(z)_convergence1}~和
~\ref{lem:NM_r(z)_convergence2}~可得
$$
q(z_0) = \left\{
\begin{array}{ll}
q_1(z_0,\widehat{\lambda}) < 1, & \mbox{if $\MCR_c
\subset \MCR_{1,1}$,} \\
q_2(z_0,\widehat{\lambda}) < 1, & \mbox{if $\MCR_c \subset
\MCR_{1,2}$},
\end{array}
\right.
$$
%and
及
\begin{equation}
\label{ineq:abs_r(zk)} |r(z_k, \lambda)| \leq \left\{
\begin{array}{ll}
q^{2^{k-1}}(z_0), & \mbox{if $\MCR_c
\subset \MCR_{1,1}$,} \\
q^{2^k-1}(z_0) \cdot r_*, & \mbox{if $\MCR_c \subset \MCR_{1,2}$},
\end{array} \ \ \ k = 1,2,\ldots, \lambda \in \MCR_c
\right.
\end{equation}
%by (\ref{eq:max_mod_1}), Lemmas \ref{lem:NM_r(z)_convergence1} and
%\ref{lem:NM_r(z)_convergence2}, where $r_* := \max\limits_{\lambda
%\in \MCR_c}|r(z_0,\lambda)|$ is a positive real independent on
%$\lambda \in \MCR_c$. It follows $\{r(z_k, \lambda)\}$ converges
%uniformly to 0 with order 2 as $k \to \infty$ for all $\lambda \in
%\MCR_c$ and that identities
其中~$r_* := \max\limits_{\lambda \in
\MCR_c}|r(z_0,\lambda)|$~为不依赖于~$\lambda \in \MCR_c$~的正实数.
于是, 对于任意的~$\lambda \in \MCR_c$, 当~$k \to
\infty$~时序列~$\{r(z_k, \lambda)\}$~一致收敛于~0,
且收敛速度是二阶的. 此外, 由关系
\begin{equation}
\label{eq:zkp} z_k^p = \frac{\lambda}{1 - r(z_k,\lambda)}, \ \ \
\lambda \in \MCR_c, \ k = 0, 1, \ldots
\end{equation}
%give the sequence $\{z_k(\lambda)\}$ is bounded uniformly for all
%$\lambda \in \MCR_c$. Thus, there is a $M > 0$, independent on $k$
%and $\lambda \in \MCR_c$, such that
可知, 对任意的~$\lambda \in \MCR_c$,
序列~$\{z_k(\lambda)\}$~是一致有界的. 因而,
存在一个不依赖于~$k$~的常数~$M > 0$~及~$\lambda \in \MCR_c$~使得
\begin{equation}
\label{ineq:abs_bounded_M} \frac{1}{p} |z_k(\lambda)| \leq M, \ \ k
\geq 0, \lambda \in \MCR_c.
\end{equation}
%By (\ref{it:NM_fun}) and (\ref{ineq:abs_bounded_M}),
由~(\ref{it:NM_fun})~及~(\ref{ineq:abs_bounded_M})~可得
\begin{align*}
|z_{k + 1}(\lambda) - z_k(\lambda)| = \frac{1}{p} |z_k(\lambda)|
|r(z_k, \lambda)| \leq M |r(z_k, \lambda)|, \ \ \ \lambda \in
\MCR_c, \ k =0, 1, \ldots,
\end{align*}
%which and (\ref{ineq:abs_r(zk)}) conclude that $\{z_{k + 1}(\lambda)
%- z_k(\lambda)\}$ is majoriant by a geometrical sequence which
%converges to zero and independent on $\lambda \in \MCR_c$. So,
%$\{z_k(\lambda)\}$ is a Cauchy sequence for each $\lambda \in
%\MCR_c$ and there is $z(\lambda)$ defined on $\MCR_c$ such that
%$z_k(\lambda)$ converges uniformly to $z(\lambda)$ for all $\lambda
%\in \MCR_c$. Let $k \to \infty$ in (\ref{eq:zkp}), we have

结合~(\ref{ineq:abs_r(zk)})~可得知序列~$\{z_{k + 1}(\lambda) -
z_k(\lambda)\}$~收敛于~0. 故对于任意的~$\lambda \in \MCR_c$,
$\{z_k(\lambda)\}$~是~Cauchy~列.
于是存在定义于~$\MCR_c$~的~$z(\lambda)$~使得对于任意的~$\lambda \in
\MCR_c$, 序列~$z_k(\lambda)$~一致收敛于~$z(\lambda)$.
在~(\ref{eq:zkp})~中令 ~$k \to \infty$~ 即可得
$$
z^p(\lambda) = \lambda, \ \ \ \lambda \in \MCR_c.
$$
%That is, $z(\lambda)$ is a $p$th root of $\lambda \in \MCR_c$.
因此, $z(\lambda)$~是~$\lambda \in \MCR_c$~的一个~$p$~次根.

%Step 2. Since for any $\lambda \in \MCR_{1,1} \bigcup \MCR_{1,2}$,
%there is a closed domain of $\MCR_{1,1}$ or $\MCR_{1,2}$ such that
%$\lambda$ belongs to it. By Step 1, $z(\lambda)$ exists for each
%$\lambda \in \MCR_{1,1}\bigcup \MCR_{1,2}$. In this step, we will
%show $z(\lambda)$ obtained in Step 1 is analytic in
%$\Int(\MCR_{1,1})$, the interior of $\MCR_{1,1}$, and $\MCR_{1,2}$.
%Therefore, $z(\lambda)$ located in a single-valued branch of root
%function in $\Int(\MCR_{1,1})$ and $\MCR_{1,2}$.

第二步. 由于对任意的~$\lambda \in \MCR_{1,1} \bigcup \MCR_{1,2}$,
存在~$\MCR_{1,1}$~ 或
~$\MCR_{1,2}$~的一个闭区域使得~$\lambda$~属于它. 由第一步知,
对于任意的~$\lambda \in \MCR_{1,1}\bigcup \MCR_{1,2}$,
$z(\lambda)$~都是存在的. 在这一步中, 将进一步证明,
$z(\lambda)$~在~$\MCR_{1,1}$~的内部~$\Int(\MCR_{1,1})$~及~$\MCR_{1,2}$~上是
解析的.
由此得到~$z(\lambda)$~落在~$\Int(\MCR_{1,1})$~和~$\MCR_{1,2}$~上的根函数的单值分枝上.


%In fact, by the definition of $z_k(\lambda), \lambda \in
%\MCR_{1,1}\bigcup\MCR_{1,2}$, we have $z_k(\lambda)$ is analytic in
%$\Int(\MCR_{1,1})$ or $\MCR_{1,2}$. Since $\{z_k(\lambda)\}$
%converges uniformly to $z(\lambda)$ by Step 1, $z_k(\lambda)$ is
%analytic in $\Int(\MCR_{1,1})$ and $\MCR_{1,2}$ by Weierstrass
%theorem, seperately. Recall that $z(\lambda)$ is a $p$th root of
%$\lambda \in \Int(\MCR_{1,1})$ or $\MCR_{1,2}$, we get $z(\lambda)$
%located in a single-valued branch of root function for $\lambda \in
%\Int(\MCR_{1,1})$ or $\lambda \in \MCR_{1,2}$, seperately.


事实上, 由~$z_k(\lambda), \lambda \in
\MCR_{1,1}\bigcup\MCR_{1,2}$~的定义知,
$z_k(\lambda)$~在~$\Int(\MCR_{1,1})$~或~$\MCR_{1,2}$~都是解析的.
又由第一步知~$\{z_k(\lambda)\}$~一致收敛于~$z(\lambda)$,
故由~Weierstrass~定理得~$z_k(\lambda)$~在~$\Int(\MCR_{1,1})$~
和~$\MCR_{1,2}$~都是解析的. 而~$z(\lambda)$~是~$\lambda \in
\Int(\MCR_{1,1})$~ 或 ~$\MCR_{1,2}$~的~$p$~次根,
因此,$z(\lambda)$~落在~$\Int(\MCR_{1,1})$~和~$\MCR_{1,2}$~上的根函数的单值分枝上.



%Step 3. In this step, we will show that $z(\lambda) \to
%z(\lambda_0)$ as $\lambda \to \lambda_0$ from the inner side of
%$\MCR_{1,1}$, where $\lambda_0 \in \partial\MCR_{1,1}$ and
%$\lambda_0 \neq 0$.

第三步. 对于~$\lambda_0 \in \partial\MCR_{1,1}$~且~$\lambda_0 \neq
0$, 将证明当~$\lambda \to
\lambda_0$~(~从~$\MCR_{1,1}$~的内部逼近)~时~$z(\lambda) \to
z(\lambda_0)$.

%It is clear that $|r(z_0, \lambda_0)| < 1$ for any $\lambda_0 \in
%\partial\MCR_{1,1}$ and $\lambda_0 \neq 0$. Then, there is $\delta_0 >
%0$ such that closed domain $\overline{O}(\lambda_0, \delta_0)
%\bigcap \MCR_{1,1}$ does not contains 0 and 1, where
%$\overline{O}(\lambda_0, \delta_0) := \{\lambda \in \CS: |\lambda -
%\lambda_0| \leq \delta_0\}$. By Step 1, $\{z_k(\lambda)\}$ converges
%uniformly to $z(\lambda)$ for any $\lambda$ in
%$\overline{O}(\lambda_0, \delta_0) \bigcap \MCR_{1,1}$. So, for any
%$\varepsilon > 0$, there exists $K > 0$ such that for all $k \geq K$
%and $\lambda \in \overline{O}(\lambda_0, \delta_0) \bigcap
%\MCR_{1,1}$,

显然, 对于任意的~$\lambda_0 \in
\partial\MCR_{1,1}$~且 ~$\lambda_0 \neq 0$, 有~$|r(z_0, \lambda_0)| <
1$. 故存在~$\delta_0 > 0$~使得闭区域~$\overline{O}(\lambda_0,
\delta_0) \bigcap \MCR_{1,1}$~不包含~0~和~1, 其中
$$
\overline{O}(\lambda_0, \delta_0) := \{\lambda \in \CS: |\lambda -
\lambda_0| \leq \delta_0\}.
$$
由第一步知, $\{z_k(\lambda)\}$~一致收敛于~$z(\lambda)$, $\lambda \in
\overline{O}(\lambda_0, \delta_0) \bigcap \MCR_{1,1}$.
故对任意的~$\varepsilon > 0$, 存在整数~$K > 0$~使得对任意的~$k \geq
K$~及~$\lambda \in \overline{O}(\lambda_0, \delta_0) \bigcap
\MCR_{1,1}$~都有
\begin{equation}
\label{ineq:NM_abs_zk-z_1} |z_k(\lambda) - z(\lambda)| <
\frac{\varepsilon}{3}.
\end{equation}
%Since $z_k(\lambda)$ is analytic in $\MCR_{1,1}$ derives
%$z_k(\lambda)$ is continuous in $\overline{O}(\lambda_0, \delta_0)
%\bigcap \MCR_{1,1}$, there exists $0 < \delta_1 < \delta_0$ such
%that $\overline{O}(\lambda_0, \delta_1) \subset
%\overline{O}(\lambda_0, \delta_0)$ and
因~$z_k(\lambda)$~在~$\MCR_{1,1}$~上是解析的,
故~$z_k(\lambda)$~在~$\overline{O}(\lambda_0, \delta_0) \bigcap
\MCR_{1,1}$~是连续的. 于是, 存在~$0 < \delta_1 <
\delta_0$~使得~$\overline{O}(\lambda_0, \delta_1) \subset
\overline{O}(\lambda_0, \delta_0)$~且
\begin{equation}
\label{ineq:NM_abs_zk-z_2} |z_k(\lambda) - z_k(\lambda_0)| <
\frac{\varepsilon}{3}, \ \ \ \forall \ \lambda \in
\overline{O}(\lambda_0, \delta_1)\bigcap \MCR_{1,1}.
\end{equation}
%Thus, for all $\lambda$ in $\overline{O}(\lambda_0, \delta_1)\bigcap
%\MCR_{1,1}$, we have
因而, 对任意的~$\lambda \in \overline{O}(\lambda_0, \delta_1)\bigcap
\MCR_{1,1}$,
由~(\ref{ineq:NM_abs_zk-z_1})~及~(\ref{ineq:NM_abs_zk-z_2})~可得
\begin{align*}
|z(\lambda) - z(\lambda_0)| & \leq |z(\lambda) - z_k(\lambda)| +
|z_k(\lambda) - z_k(\lambda_0)| + |z_k(\lambda_0) - z(\lambda_0)|\\
& < \frac{\varepsilon}{3} + \frac{\varepsilon}{3} +
\frac{\varepsilon}{3} = \varepsilon
\end{align*}
%by (\ref{ineq:NM_abs_zk-z_1}) and (\ref{ineq:NM_abs_zk-z_2}). That is,
%$z(\lambda)$ is continuous at $\lambda = \lambda_0$. The
%arbitrariness of $\lambda_0$ completes the proof of Step 3.
即~$z(\lambda)$~在~$\lambda = \lambda_0$~是连续的.
由~$\lambda_0$~的任意性即得证.


%Step 4. This is the last step of the proof.



%By Step 2, $z(\lambda)$ is analytic in $\Int(\MCR_{1,1})$ and it is
%located in a single-valued branch of $p$th root function. Since
%$z_k(1) \equiv 1$ for all $k$ implies $z(1) = 1$, we have that
%$z(\lambda)$ located in the single-valued branch containing 1. That
%is, $z(\lambda)$ is the principal $p$th root of each $\lambda$ in
%$\Int(\MCR_{1,1})$. Since $z(\lambda) \to z(\lambda_0)$ as $\lambda
%\to \lambda_0 \ (\lambda \in \Int(\MCR_{1,1}))$ for each $\lambda_0
%\in
%\partial\MCR_{1,1}$ by Step 3, we get $z(\lambda_0)$ is also the principal $p$th
%root of $\lambda_0 \in \partial\MCR_{1,1}$.


第四步.
因~$z(\lambda)$~在~$\Int(\MCR_{1,1})$~是解析的且落在~$p$~次根的单值分支上,
又因对任意的~$k\geq0$~都有~$z_k(1) \equiv 1$, 故~$z(1) =
1$~且~$z(\lambda)$~落在包含~1~的单值分支中. 即对任意~$\lambda \in
\Int(\MCR_{1,1})$, $z(\lambda)$~都是其主~$p$~次根. 注意到,
对任意的~$\lambda_0 \in
\partial\MCR_{1,1}$, 当~$\lambda
\to \lambda_0 \ (\lambda \in \Int(\MCR_{1,1}))$~时~$z(\lambda) \to
z(\lambda_0)$, 因而可知~$z(\lambda_0)$~亦是~$\lambda_0 \in
\partial\MCR_{1,1}$~的主~$p$~次根.


%By now, we have proved that $z(\lambda)$ is the principal $p$th root
%of each $\lambda \in \MCR_{1,1}$ except $\lambda = 0$. When $\lambda
%= 0$, we have by (\ref{it:NM}) that

此时, 我们已经证明当~$\lambda \in \MCR_{1,1}$~且~$\lambda \neq
0$~时~$z(\lambda)$~是其主~$p$~次根. 现在考虑当~$\lambda =
0$~时的情形. 由~(\ref{it:NM})~可得
$$
z_k(0) = \frac{p-1}{p} z_{k-1}(0) = \left(\frac{p-1}{p}\right)^k
z_0, \ \ k = 0, 1, 2, \ldots.
$$
%So, $\{z_k(0)\}$ converges to 0, the principal $p$th root of 0,
%linearly. Therefore, $z(\lambda)$ is the principal $p$th root of
%each $\lambda \in \MCR_{1,1}$.
故 ~$\{z_k(0)\}$~线性收敛到~0~(亦即其自身的主~$p$~次根). 因此,
对任意的~$\lambda \in \MCR_{1,1}$, $z(\lambda)$~都是其主~$p$~次根.

%By Step 2 again, $z(\lambda)$ is analytic and locates in a
%single-valued branch of $p$th root function for each $\lambda \in
%\MCR_{1,2}$. Now, we can see that, if we can show $\MCR_{1,2}$
%contains some part of $\partial\MCR_{1,1}$, then the single-valued
%branch of $z(\lambda)$ for $\lambda \in \MCR_{1,2}$ is the same
%branch of $z(\lambda)$ for $\lambda \in \MCR_{1,1}$, which deduces
%that $z(\lambda)$ is the principle $p$th root of $\lambda \in
%\MCR_{1,1} \bigcup\MCR_{1,2}$. So, what we need to do is to prove
%$\MCR_{1,2}$ contains some part of $\partial\MCR_{1,1}$.

同样由第二步知, 对任意的~$\lambda \in \MCR_{1,2}$,
$z(\lambda)$~是解析的且落在~$p$~次根函数的单值分支中. 于是,
若我们能证明~$\MCR_{1,2}$~包含~$\partial\MCR_{1,1}$~某一部分,
那么可知, 当~$\lambda \in
\MCR_{1,2}$~时的~$z(\lambda)$~的单值分支与当~$\lambda \in
\MCR_{1,1}$~时的~$z(\lambda)$~的单值分支是一样的. 从而可得,
对任意~$\lambda \in \MCR_{1,1} \bigcup\MCR_{1,2}$,
有~$z(\lambda)$~为其主~$p$~次根. 因此,
我们只须证明~$\MCR_{1,2}$~包含了~$\partial\MCR_{1,1}$~的某一部分即可.

%Set $\lambda_0 = z_0^p$ and define

记~$\lambda_0 = z_0^p$. 定义
\begin{align*}
S_{1,m}(z_0,\lambda) & := \sum\limits_{j=2}^m
c_{1,j}r^{j-1}(z_0,\lambda), \\
S_\infty(z_0,\lambda) & := \sum\limits_{j=2}^\infty
c_{1,j}r^{j-1}(z_0,\lambda), \quad m \geq 2, \lambda \in
\partial\MCR_{1,1}.
\end{align*}
%Clearly, $S_\infty(z_0,\lambda)$ is continuous with respect to
%$\lambda$ on $\partial\MCR_{1,1}$ and
显然, $S_\infty(z_0,\lambda)$~关于~$\lambda$~是连续的且
\begin{equation}
\label{ineq:S(z0,lambda0)} 0 = \left|S_\infty(z_0,\lambda_0)\right|
= \min_{\lambda \in
\partial\MCR_{1,1}}\left|S_\infty(z_0,\lambda)\right| \leq
\left|S_\infty(z_0,\lambda)\right| \leq 1, \ \ \ \forall \ \lambda
\in
\partial\MCR_{1,1}.
\end{equation}
%Since $|S_{1,m}(z_0,\lambda)|<1$ holds for all $\lambda \in
%\MCR_{1,1}\bigcup\MCR_{1,2}$, we can choose $M < 1$ be a real
%satisfying
由于对任意的~$\lambda \in
\MCR_{1,1}\bigcup\MCR_{1,2}$~都有~$|S_{1,m}(z_0,\lambda)|<1$,
故可取实数~$M < 1$~满足不等式
$$
\sup_{m\geq2}\left|S_{1,m}(z_0,\lambda)\right| < M, \ \ \ \lambda
\in
\partial\MCR_{1,1}.
$$
%By (\ref{ineq:S(z0,lambda0)}), there is $\delta > 0$ such that once
%$|\lambda - \lambda_0| < \delta$, then
由~(\ref{ineq:S(z0,lambda0)})~知, 存在~$\delta >
0$~使得一旦~$|\lambda - \lambda_0| < \delta$~便有
$$
 \left|S_\infty(z_0,\lambda)\right| <
 \frac{\frac{1}{M}-1}{\frac{1}{M} + 1},
$$
%or equivalently,
或其等价情形
$$
q_2(z_0,\lambda) = \sup_{m\geq 2}|S_{1,m}(z_0,\lambda)|\cdot
\frac{1+|S_\infty(z_0,\lambda)|}{1-|S_\infty(z_0,\lambda)|} < M
\cdot \frac{1}{M} = 1.
$$
%Therefore, $\MCR_{1,2}$ contains some part of $\partial\MCR_{1,1}$.
%The proof is completed.
因此, $\MCR_{1,2}$~包含了~$\partial\MCR_{1,1}$~的某一部分. 证完.
\end{proof}


%
%Choosing $z_0 \equiv 1$ in $\MCR_{1,1}$ and $\MCR_{1,2}$ given by
%(\ref{set:R11}) and (\ref{set:R12}), respectively, one has that
%$\MCR = \MCR_{1,1} \bigcup \MCR_{1,2}$. So, we obtain from Lemma
%\ref{lem:ScalarNMCon1} the following corollary:


对于分别由~(\ref{set:R11})~和~(\ref{set:R12})~定义的~$\MCR_{1,1}$~和
~$\MCR_{1,2}$, 若取~$z_0 \equiv 1$, 则有~$\MCR = \MCR_{1,1} \bigcup
\MCR_{1,2}$. 于是, 由引理~\ref{lem:ScalarNMCon1}~可立即得到如下推论:


\begin{corollary}
\label{cor:NM_zk_convergence1} %For any $\lambda \in \MCR_1$ defined
%in $(\ref{set:R_NM})$, the sequence $\{z_k(\lambda)\}$ generated by
%scalar Newton iteration $(\ref{it:NM})$ with $z_0 = 1$ for solving
%$(\ref{fun:f(z)})$ converges to the principal $p$th root
%$\lambda^{1/p}$. Moreover, if $\lambda \neq 0$, then the convergence
%order is $2$.
%
对于任意的~$\lambda \in \MCR_1$,
其中~$\MCR_1$~由~$(\ref{set:R_NM})$~定义, 由标量
~Newton~法~$(\ref{it:NM})$~以~$z_0 =
1$~为初始点进行迭代产生的序列~$\{z_k(\lambda)\}$~
收敛于~$\lambda$~的主~$p$~次根~$\lambda^{1/p}$. 进一步, 若~$\lambda
\neq 0$, 则其收敛速度是二阶的.
\end{corollary}





%Next, we consider the case of matrix form. For the matrix $A \in
%\CS^{n \times n}$ defined in (\ref{eq:f(X)=0}), let

下面我们考虑矩阵的情形. 对于由~(\ref{eq:f(X)=0})~给定的矩阵~$A \in
\CS^{n \times n}$, 令
\begin{equation}
\label{fun:NM_R(X)} R(X) := \I - A X^{-p}, \ \ \ p \geq 2,
\end{equation}
%for any nonsingular matrix $ X \in \CS^{n \times n}$.
其中~$ X \in \CS^{n \times n}$~是非奇异的.

\begin{lemma}
\label{lem:R(N(X))_R(X)} %Let $X \in \CS^{n\times n}$ be a
%nonsingular matrix commuting with $A$ and $R(X)$ be defined in
%$(\ref{fun:NM_R(X)})$. If the spectrum $\sigma(R(X)) \subset
%\MCD_{0,1}$ for some nonsingular matrix $X \in \CS^{n \times n}$,
%where $\MCD_{0,1}$ is defined in $(\ref{set:D0_NM})$, then $N(X)$
%generated by $(\ref{it:NM_fun})$ is nonsingular, commutes with $A$
%and
%
设矩阵~$X \in \CS^{n\times n}$~是非奇异的且与~$A$~是可交换的,
$R(X)$~由~$(\ref{fun:NM_R(X)})$~定义.
如果~$R(X)$~的谱满足~$\sigma(R(X)) \subset \MCD_{0,1}$,
其中~$\MCD_{0,1}$~由~$(\ref{set:D0_NM})$~定义,
那么由~$(\ref{it:NM_fun})$~定义~$N(X)$~是非奇异的,
与~$A$~是可交换的且满足
\begin{equation}
\label{eq:R(N(X))} R(N(X)) = \I - \left[\I - \frac{1}{p}
R(X)\right]^{-p}\cdot(\I - R(X)) = \phi_1(R(X)),
\end{equation}
%where $\phi_1$ is defined in $(\ref{fun:phi(z)_NM})$.
%
其中~$\phi_1$~由~$(\ref{fun:phi(z)_NM})$~定义.
\end{lemma}

\begin{proof}
%Clearly, $N(X)$ given by (\ref{it:NM_fun}) exists for any
%nonsingular matrix $X \in \CS^{n \times n}$. Since $\sigma(R(X))
%\subset \MCD_{0,1}\backslash\{0\}$, we get $ \I - \frac{1}{p} R(X)$
%is nonsingular by Neumann Lemma. It follows that

显然, 若~$X \in \CS^{n \times n}$~是非奇异矩阵,
则由~(\ref{it:NM_fun})~定义的~$N(X)$~是存在的. 由于~$\sigma(R(X))
\subset \MCD_{0,1}\backslash\{0\}$, 根据~Neumann~引理可知矩阵
$$
\I - \frac{1}{p} R(X)
$$
是非奇异的. 于是由等式
\begin{align*}
N(X) = \frac{1}{p} X \left[(p - 1)\I + AX^{-p}\right]
 = X \left[\I - \frac{1}{p}R(X)\right]
\end{align*}
%is also nonsingular and commutes with $A$, and
%
知矩阵~$N(X)$~是非奇异的且与~$A$~是可交换的. 由此可进一步得到
\begin{equation*}
R(N(X)) = \I - \left[\I - \frac{1}{p} R(X)\right]^{-p}\cdot(\I -
R(X)) = \phi_1(R(X))
\end{equation*}
%by the assumption of $X$ commutes with $A$. This completes the
%proof.
%
上述第一个等式是根据~$X$~与~$A$~是可交换的假设得到的. 证完.
\end{proof}



%Thanks to Lemma \ref{lem:R(N(X))_R(X)}, we have

根据引理~\ref{lem:R(N(X))_R(X)}~可直接得到如下推论:

\begin{corollary}
\label{cor:N(Xk)} %If $X_0 \in \CS^{n \times n}$ commutes with $A$
%and the spectrum $\sigma(R(X_0)) \subset \MCD_{0,1}$, where
%$\MCD_{0,1}$ is defined in $(\ref{set:D0_NM})$, then the sequence
%$\{X_k\}$ starting from $X_0$ generated by Newton's method
%$(\ref{it:NM})$ for solving $(\ref{eq:f(X)=0})$ exists.
%
如果矩阵~$X_0 \in \CS^{n \times
n}$~与~$A$~是可交换的且~$R(X_0)$~的谱满足~$\sigma(R(X_0)) \subset
\MCD_{0,1}$, 其中~$\MCD_{0,1}$~由~$(\ref{set:D0_NM})$~定义,
那么由~Newton~法~$(\ref{it:NM})$~以~$X_0$~为
初始点进行迭代产生的矩阵序列~$\{X_k\}$~是有定义的.
\end{corollary}



\begin{lemma}
\label{lem:norm_R(N(X))_R(X)} %If the spectrum $\sigma(R(X)) \subset
%\MCD_1\backslash\{0\}$ for some nonsingular matrix $X \in \CS^{n
%\times n}$, where $R(X)$ be defined in $(\ref{fun:NM_R(X)})$ and
%$\MCD_1$ is defined in $(\ref{set:D1_NM})$, then, there is a
%sub-multiplicative matrix norm $\|\cdot\|$ such that $\|R(X)\| \leq
%1$ and
%
设~$R(X)$~由~$(\ref{fun:NM_R(X)})$~定义, $X \in \CS^{n \times
n}$~是非奇异的. 如果~$R(X)$~的谱满足~$\sigma(R(X)) \subset
\MCD_1\backslash\{0\}$, 其中~$\MCD_1$~由~$(\ref{set:D1_NM})$~定义,
那么存在一个次可加的矩阵范数~$\|\cdot\|$~使得~$\|R(X)\| \leq 1$~且
\begin{equation}
\label{ineq:NM_norm_R(E(X))_1} \|R(E(X))\| \leq
\frac{\phi_1(\|R(X)\|)}{\|R(X)\|^2} \cdot \|R(X)\|^2 < \|R(X)\|^2,
\end{equation}
%where $\phi_1$ is defined by $(\ref{fun:phi(z)_NM})$.
%
其中~$\phi_1$~由~$(\ref{fun:phi(z)_NM})$~定义.
\end{lemma}

\begin{proof}
%It follows from Lemma \ref{lem:R(N(X))_R(X)} that $R(N(X))$ exists.
%Since the spectrum $\sigma(R(X)) \subset \MCD_1$, the spectral
%radius of $R(X)$ is less than 1. So, there is a sub-multiplicative
%matrix norm $\|\cdot\|$ such that $\|R(X)\| < 1$. Note that, for any
%$u \in (0,1)$, It follows from that
%
由引理~\ref{lem:R(N(X))_R(X)}~易知~$R(N(X))$~是存在的.
因~$\sigma(R(X)) \subset \MCD_1$, 故~$R(X)$~的谱半径小于~1.
从而存在一个次可加的矩阵范数~$\|\cdot\|$~使得~$\|R(X)\| < 1$.
注意到, 对于任意的~$u \in (0,1)$~都有
$$
\frac{\phi(u)}{u^2} = \sum_{j=2}^\infty c_{1,j} u^{j-2} <
\sum_{j=2}^\infty c_{1,j} = 1.
$$
%Thus, this together with (\ref{eq:R(N(X))}) gives that
%
于是, 结合~(\ref{eq:R(N(X))})~可得
\begin{align*}
\|R(N(X))\| = \|\phi_1(R(X))\| & \leq \phi_1(\|R(X)\|) \\
& = \frac{\phi_1(\|R(X)\|)}{\|R(X)\|^2}\|R(X)\|^2 \\
& < \|R(X)\|^2,
\end{align*}
%which shows (\ref{ineq:NM_norm_R(E(X))_1}). The proof is completed.
%
因此~(\ref{ineq:NM_norm_R(E(X))_1})~是成立的. 证完.
\end{proof}



\begin{lemma}
\label{lem:NM_R(X)_convergence1} %Let $R(X)$ be defined in
%$(\ref{fun:NM_R(X)})$. Suppose the spectrum $\sigma(R(X_0)) \subset
%\mathcal {D}_1\backslash\{0\}$ for some nonsingular matrix $X_0 \in
%\CS^{n \times n}$ and that $\|R(X_0)\| < 1$ for a sub-multiplicative
%matrix norm $\|\cdot\|$, where $\mathcal {D}_1$ is defined in
%$(\ref{set:D1_NM})$. Let $\{X_k\}$ be the sequence starting from
%$X_0$ generated by Newton's method $(\ref{it:NM})$ for solving
%$(\ref{eq:f(X)=0})$. Then we have
%
设~$R(X)$~由~$(\ref{fun:NM_R(X)})$~定义, 矩阵~$X_0 \in \CS^{n \times
n}$~是非奇异的. 设~$R(X_0)$~的谱满足~$\sigma(R(X_0)) \subset
\mathcal
{D}_1\backslash\{0\}$~且存在一个次可加性的矩阵范数~$\|\cdot\|$~使得~$\|R(X_0)\|
< 1$, 其中~$\mathcal {D}_1$~由~$(\ref{set:D1_NM})$~定义.
令~$\{X_k\}$~是由~Newton~法~$(\ref{it:NM})$~以~$X_0$~为初始点进行迭代产生的矩阵序列.
则有
\begin{equation}
\label{ineq:NM_norm_R(Xk)_1} \|R(X_k)\| \leq q^{2^k-1}(X_0) \cdot
\|R(X_0)\|, \ \ \ k = 1, 2, \ldots,
\end{equation}
%where
其中
\begin{equation}
\label{cons:NM_q(X0)} q(X_0) := \frac{\phi(\|R(X_0)\|)}{\|R(X_0)\|}
< 1,
\end{equation}
%and $\phi_1$ is defined by $(\ref{fun:phi(z)_NM})$.
%
而~$\phi_1$~由~$(\ref{fun:phi(z)_NM})$~定义.
\end{lemma}

\begin{proof}
%For $X_0$ chosen, by (\ref{ineq:NM_norm_R(E(X))_1}) in Lemma
%\ref{lem:norm_R(N(X))_R(X)}, we have

对于取定的矩阵~$X_0$,
由引理~\ref{lem:norm_R(N(X))_R(X)}~中的~(\ref{ineq:NM_norm_R(E(X))_1})~可得
$$
\|R(X_1)\| \leq \frac{\phi_1(\|R(X_0)\|)}{\|R(X_0)\|^2} \|R(X_0)\|^2
= q(X_0) \cdot \|R(X_0)\|.
$$
%If (\ref{ineq:NM_norm_R(Xk)_1}) holds for some $k \geq 1$, then by
%Lemma \ref{lem:norm_R(N(X))_R(X)} again, one has that
%
如果对于某个整数~$k \geq 1$, (\ref{ineq:NM_norm_R(Xk)_1})~是成立的,
那么由引理~\ref{lem:norm_R(N(X))_R(X)}~有
\begin{align*}
\|R(X_{k+1})\| & \leq
\frac{\phi_1(\|R(X_k)\|)}{\|R(X_k)\|^2}\|R(X_k)\|^2\\
& \leq \frac{\phi_1(\|R(X_0)\|)}{\|R(X_0)\|^2}
\left[q^{2^k-1}(X_0)\right]^2 \cdot \|R(X_0)\|^2\\
& = [q(X_0)]^{2^{k+1}-1}\cdot\|R(X_0)\|.
\end{align*}
%Thus, by induction, (\ref{ineq:NM_norm_R(Xk)_1}) holds for all $k
%\geq 1$. This completes the proof.
%
因而, 由归纳法知, 对于任意的~$k \geq 1$,
(\ref{ineq:NM_norm_R(Xk)_1})~都是成立的. 证完.
\end{proof}



\begin{lemma}
\label{lem:NM_R(X)_convergence2} %Let $R(X)$ be defined in
%$(\ref{fun:NM_R(X)})$. If the spectrum $\sigma(R(X_0)) \subset
%\MCD_{2,1} \bigcap \MCD_{0,1}\backslash\{0\}$ for some nonsingular
%matrix $X_0 \in \CS^{n \times n}$, where $\MCD_{0,1}$ and
%$\MCD_{2,1}$ are defined in $(\ref{set:D0_NM})$ and
%$(\ref{set:D2_NM})$, respectively. Let $\{X_k\}$ be the sequence
%starting from $X_0$ generated by Newton's method $(\ref{it:NM})$ for
%solving $(\ref{eq:f(X)=0})$. Then, there exists $\widehat{N} > 0$
%such that
%
设~$R(X)$~由~$(\ref{fun:NM_R(X)})$~定义, 矩阵~$X_0 \in \CS^{n \times
n}$~是非奇异的. 如果~$R(X_0)$~的谱满足~$\sigma(R(X_0)) \subset
\MCD_{2,1} \bigcap \MCD_{0,1}\backslash\{0\}$, 其中~$\MCD_{0,1}$~和
~$\MCD_{2,1}$~分别由~$(\ref{set:D0_NM})$~和~$(\ref{set:D2_NM})$~定义.
设~$\{X_k\}$~是由~Newton~法~$(\ref{it:NM})$~以~$X_0$~为初始点进行迭代产生的矩阵序列.
则存在整数~$\widehat{N} > 0$~使得
\begin{equation}
\label{ineq:NM_norm_R(Xk)_2} \|R(X_k)\| \leq
[q(X_{\widehat{N}})]^{2^{k-\widehat{N}}-1} \cdot
\|R(X_{\widehat{N}})\|, \quad \forall \ k
> \widehat{N},
\end{equation}
%where $q(X_{\widehat{N}})$ is defined as $q(X_0)$ in
%$(\ref{cons:NM_q(X0)})$ by substituting $X_{\widehat{N}}$ for $X_0$.
%
其中
\begin{equation*}
q(X_{\widehat{N}}) :=
\frac{\phi(\|R(X_{\widehat{N}})\|)}{\|R(X_{\widehat{N}})\|} < 1,
\end{equation*}
而~$\phi_1$~由~$(\ref{fun:phi(z)_NM})$~定义.
\end{lemma}

\begin{proof}
%For any $r(z_0) \in \sigma(R(X_0))$, let $\{z_k\}$ be the sequence
% generated by the scalar form of (\ref{it:NM}) starting from $z_0$. It
%follows from Lemma \ref{lem:NM_r(z)_convergence2} that there exists
%$N
%> 0$ such that $|r(z_N)| < 1$. Then, we can get from Lemma
%\ref{lem:NM_r(z)_convergence1} that

对于任意的~$r(z_0) \in \sigma(R(X_0))$,
设~$\{z_k\}$~是由标量~Newton~法~(\ref{it:NM})~以~$z_0$~为初始点进行迭代产生的复序列.
则由引理~\ref{lem:NM_r(z)_convergence2}~知, 存在整数~$N
> 0$~使得~$|r(z_N)| < 1$. 于是,
由引理~\ref{lem:NM_r(z)_convergence1}~可得
\begin{equation*}
\label{ineq:abs_r(zk)_3} |r(z_k)| < |r(z_N)|^{2^{k-N}}, \quad
\forall \ k > N.
\end{equation*}
%Define
定义
$$
\widehat{N} := \max_{r(z_0) \in \sigma(R(X_0))} \{N: \text{choose a
}N
> 0 \text{ such that } |r(z_N)| < 1\}.
$$
%Then, there exists a sub-multiplicative matrix norm $\|\cdot\|$ such
%that $\|X_{\widehat{N}}\| < 1$. Thus, (\ref{ineq:NM_norm_R(Xk)_2})
%follows from Lemma \ref{lem:NM_R(X)_convergence1}. This completes
%the proof.
%
则存在一个次可加性的矩阵范数~$\|\cdot\|$~使得~$\|X_{\widehat{N}}\| <
1$. 于是,
由引理~\ref{lem:NM_R(X)_convergence1}~可知~(\ref{ineq:NM_norm_R(Xk)_2})~是成立的.
证完.
\end{proof}



\section{定理~\ref{th:MatrixNMCon}~的证明}


%The following lemma is taken from \cite[Theorem 4.15]{Higham2008},
%which allows us to deduce convergence of the matrix iteration
%sequence generated by Newton's method (\ref{it:NM}).

下面的引理来自~\cite[定理~4.15]{Higham2008}.
根据该引理并结合上节中的引理即可证明我们的收敛定理.

\begin{lemma}[{\cite[定理~4.15]{Higham2008}}]
\label{lem:MatIteConLem} %Suppose that $g(x,t)$ is a rational
%function with respect to its two variables and that $x^* =
%f(\lambda)$ is an attracting fixed point of the iteration $x_{k + 1}
%= g(x_k, \lambda), x_0 = \phi_0(\lambda)$, where $\phi_0$ is a
%rational function and $\lambda \in \CS$. Then, the matrix sequence
%generated by $X_{k + 1} = g(X_k, J(\lambda)), X_0 =
%\phi_0(J(\lambda))$, converges to a matrix $X^*$ with $(X^*)_{ii}
%\equiv f(\lambda)$, $i = 1, 2, \ldots, m$, where $J(\lambda) \in
%\CS^{m \times m}$ is a Jordan block.

设~$g(x,t)$~是一个双变量的有理函数, $x^* =
f(\lambda)$~是如下迭代格式的一个吸引固定点:
$$
x_{k + 1} = g(x_k, \lambda), \quad x_0 = \phi_0(\lambda),
$$
其中 ~$\phi_0$~是一个有理函数而~$\lambda \in \CS$. 则由迭代格式
$$
X_{k + 1} = g(X_k, J(\lambda)), \quad X_0 = \phi_0(J(\lambda))
$$
迭代产生的矩阵序列~$\{X_k\}$~收敛于满足如下关系的矩阵~$X^*$:
$$
(X^*)_{ii} \equiv f(\lambda), \quad i = 1, 2, \ldots, m,
$$
其中 ~$J(\lambda) \in \CS^{m \times m}$~为~Jordan~块.
\end{lemma}




%Now, we can prove Theorem \ref{th:MatrixNMCon} by applying the above
%lemmas.

下面应用引理~\ref{lem:MatIteConLem}~来证明收敛性定理~\ref{th:MatrixNMCon}.

\begin{proof}[定理~$\ref{th:MatrixNMCon}$~的证明]
%By Corollary \ref{cor:N(Xk)}, the matrix sequence $\{X_k\}$
%generated by Newton's method (\ref{it:NM}) starting from $X_0 = \I$
%is well defined when the eigenvalues of $A$ are in $\MCR_1 \subset
%\MCD_{0,1}$. Thanks to Lemma \ref{lem:MatIteConLem}, $\{X_k\}$
%converges to the principal $p$th root of $A$ follows from Corollary
%\ref{cor:NM_zk_convergence1}. The first part of theorem is
%completed.

首先由推论~\ref{cor:N(Xk)}~知, 当矩阵~$A$~的所有特征值都属于~$\MCR_1
\subset \MCD_{0,1}$, 则由~Newton~法~(\ref{it:NM})~以~$X_0 =
\I$~为初始点进行迭代产生的矩阵序列~$\{X_k\}$~是有定义的.
再由引理~\ref{lem:MatIteConLem}~及推论~\ref{cor:NM_zk_convergence1}~知,
矩阵序列~$\{X_k\}$~收敛于矩阵~$A$~的主~$p$~次根~$A^{1/p}$.

%For the second part, let $X_* = A^{1/p}$ and $E_k = X_k - X_*$ for
%$k \geq 0$. Due to the commutativity of $X_k$ and $X_*$, we have

对于~$k \geq 0$, 令~$X_* = A^{1/p}$~及~$E_k = X_k - X_*$.
由于~$X_k$~与~$X_*$~是可交换的, 则有
\begin{align}
R(X_k) & = \I - AX_k^{-p} = (X_k^p - X_*^p)X_k^{-p} \nonumber\\
& = (X_k - X_*)\left(X_k^{p - 1} + X_k^{p - 2}X_* + \cdots +
X_k X_*^{p - 2} + X_*^{p - 1}\right)X_k^{-p} \nonumber\\
& = E_k \left(X_k^{p - 1} + X_k^{p - 2}X_* + \cdots + X_k X_*^{p -
2} + X_*^{p - 1}\right) X_k^{-p}, \ \ \forall \ k \geq 0
\label{eq:R(Xk)}
\end{align}
%Set
记
$$
Y_k := \sum_{i = 1}^p X_k^{p - i} X_*^{i - 1} = X_k^{p - 1} + X_k^{p
- 2}X_* + \cdots + X_k X_*^{p - 2} + X_*^{p - 1}.
$$
%Since $X_k$ converges to $A^{1/p}$ and all the eigenvalues of
%$A^{1/p}$ are not in $\RS^-$, there exists nonnegative integer $N >
%0$ such that the eigenvalues $X_k$ are not in $\RS^-$ for all $k
%\geq N$. Thus, the eigenvalues of $Y_k$ are also not in $\RS^-$ and
%so $Y_k$ is nonsingular for $k \geq N$. Then, it follows from
%(\ref{eq:R(Xk)}) that
%
因~$X_k$~收敛于~$A^{1/p}$~且~$A^{1/p}$~的所有特征值均不属于~$\RS^-$,
故存在非负整数~$N > 0$~使得对于任意的~$k \geq N$,
$X_k$~的特征值均不属于~$\RS^-$. 进而知~$Y_k$~的特征值亦不在~$\RS^-$,
从而对任意的~$k \geq N$~矩阵~$Y_k$~都是非奇异的.
于是由~(\ref{eq:R(Xk)})~可得
\begin{equation}
\label{eq:Ek+1} E_{k + 1} = R(X_{k + 1}) X_{k + 1}^p Y_{k + 1}^{-1},
\ \ k \geq N.
\end{equation}
%Thanks to (\ref{ineq:NM_norm_R(Xk)_1}) and
%(\ref{ineq:NM_norm_R(Xk)_2}) in Lemmas
%\ref{lem:NM_R(X)_convergence1} and \ref{lem:NM_R(X)_convergence2},
%respectively, there exists $K_0 > 0$ such that $\|R(X_{k + 1})\| <
%\|R(X_{k})\|^2$ for any $k \geq K_0$. It follows from
%(\ref{eq:R(Xk)}) and (\ref{eq:Ek+1}) that
%
由~(\ref{ineq:NM_norm_R(Xk)_1})~和~(\ref{ineq:NM_norm_R(Xk)_2})~可知,
存在~$K_0 > 0$~使得对于所有的~$k \geq K_0$~都有~$\|R(X_{k + 1})\| <
\|R(X_{k})\|^2$. 所以, 应用~(\ref{eq:R(Xk)})~和~(\ref{eq:Ek+1})~可得
\begin{align}
\|E_{k + 1}\| & \leq \|R(X_{k + 1})\| \|X_{k + 1}\|^p \|Y_{k + 1}^{-1}\| \nonumber\\
& < \|R(X_k)\|^2 \|X_{k + 1}\|^p \|Y_{k + 1}^{-1}\| \nonumber\\
& \leq \left(\|X_k^{-1}\|^p\|X_{k + 1}\|^p \|Y_k\|\|Y_{k +
1}^{-1}\|\right) \|E_k\|^2, \quad \forall \ k \geq K_0.
\label{ineq_norm_Ek+1}
\end{align}
%Thus $\{X_k\}$ is convergent guarantees that $\|X_k^{-1}\|^p\|X_{k +
%1}\|^p \|Y_k\|\|Y_{k + 1}^{-1}\|$ is bounded for all $k \geq K_0$.
%(\ref{ineq_norm_Ek+1}) concludes that the local convergence order is
%2. This completes the proof.
%
注意到因为~$\{X_k\}$~是收敛的, 所以对于所有的~$k \geq K_0$,
$\|X_k^{-1}\|^p\|X_{k + 1}\|^p \|Y_k\|\|Y_{k + 1}^{-1}\|$~是有界的.
因此, 由~(\ref{ineq_norm_Ek+1})~知收敛速度是二阶的. 证完.
\end{proof}






\section{Numerical examples}

Newton's method (\ref{it:NM}) and Halley's method (\ref{it:HM}) are
usually not stable in a neighborhood of the principal $p$th root of
$A$, see \cite{Iannazzo2006} or \cite{Smith2003} for the stability
analysis on Newton's method. Thus, these two iterative methods
cannot be used directly to computing $A^{1/p}$. A stable version of
Newton's method by introducing the auxiliary matrix $N_k$ has been
given in \cite{Iannazzo2006} as follows:
\begin{equation}
\label{it:CoupledNM} \left\{
\begin{array}{ll}
\displaystyle X_{k + 1} = X_k \left(\frac{(p-1)\I + N_k}{p}\right),
& X_0 =
\I, \\
\displaystyle N_{k + 1} = \left(\frac{(p-1)\I + N_k}{p}\right)^{- p}
N_k, & N_0 = A.
\end{array} \right.
\end{equation}
Clearly, $N_k = AX_k^{-p}$ and $\{X_k\}$ generated by
(\ref{it:CoupledNM}) is same as the sequence of Newton's method
(\ref{it:NM}). We call it coupled Newton iteration. When the
sequence $\{X_k\}$ generated by (\ref{it:CoupledNM}) converges to
$A^{1/p}$, $N_k$ converges to $\I$.

For Halley's method, a stable version has been given in
\cite{Iannazzo2008} as follows:
\begin{equation}
\label{it:CoupledHM} \left\{
\begin{array}{ll}
\displaystyle X_{k + 1} = X_k \big((p+1)\I + (p-1)N_k\big)^{-1}
\big((p-1)\I + (p+1)N_k\big), & X_0 =
\I, \\
\displaystyle N_{k + 1} = N_k \left(\big((p+1)\I +
(p-1)N_k\big)^{-1} \big((p-1)\I + (p+1)N_k\big)\right)^{- p}, & N_0
= A.
\end{array} \right.
\end{equation}
Also, $\{X_k\}$ generated by (\ref{it:CoupledHM}) is same as the
sequence of Halley's method (\ref{it:HM}). We call it coupled Halley
iteration. When the sequence $\{X_k\}$ generated by
(\ref{it:CoupledHM}) converges to $A^{1/p}$, $N_k$ converges to
$\I$.


\begin{algorithm}
\floatname{algorithm}{算法}
\caption{Preprocessing iterative
framework for computing $A^{1/p}$} \label{al:SIM} Given $A \in
\CS^{n \times n}$ with no nonpositive real eigenvalues, an integer
$p = 2^{k_0}q$ with $k_0 \geq 0$ and $q$ odd. This algorithm
computes the principal $p$th root of matrix $A$ via a Schur
decomposition and some iterative method.
%\newcounter{newlist}
\begin{list}{\arabic{newlist}.}{\usecounter{newlist}
\setlength{\rightmargin}{0em}\setlength{\leftmargin}{1.2em}}
\item
Compute the Schur decomposition of $A = QRQ^*$;
\item
If $q = 1$, then $k_1 = k_0$; else choose the smallest $k_1 \geq
k_0$ such that for each eigenvalue $\lambda$ of matrix $A$,
$\lambda^{1/2^{k_1}}$ belongs to some region $\MCR$;
\item
Compute $B = R^{1/2^{k_1}}$ by taking the square root $k_1$ times;
if $q = 1$, then set $X = QB\tran{Q}$; else continue;
\item
Compute $C = B^{1/q}$ by using some iterative method and set $X = Q
C^{2^{k_1 - k_0}} \tran{Q}$.
\end{list}
\end{algorithm}

To improve the convergence of iterative methods for computing
$A^{1/p}$, an effective way is of using a preprocessing as in
\cite{GuoHigham2006} so that the eigenvalues of the new matrix are
in a smaller convergence region. Based on this way, a framework that
computes the principal $p$th root of matrix $A \in \CS^{n \times n}$
by using Schur decomposition and some iterative method is summarized
as Algorithm
%\ref{al:SIM}.
A number of modifications to this
framework are possible. For example, we can obtain an algorithm
called Schur-Newton algorithm by choosing $\MCR_1^{\text{N}}$
defined in (\ref{set:R_N_pra}) as the region $\MCR$ in step 2 and
the coupled Newton iteration (\ref{it:CoupledNM}) as the iterative
method in step 4.



\begin{remark}
In practice, it is not feasible to check whether a eigenvalue
$\lambda$ belongs to $\MCR_1$. This is due to the computational cost
of (\ref{set:D2_NM}) may be large even if we only choose $m=20$.
Thus, based on the observation from Figure \ref{fig:NM_ConvReg}, we
now give a new convergence region which allows us to determine
easily whether a eigenvalue belongs to it. Define
\begin{equation}
\label{set:R_N_pra} \MCR_1^{\text{N}} = \MCD_3 \bigcup \MCD_4,
\end{equation}
where
\begin{align}
\MCD_3 & := \{z \in \CS: 1-z \in \overline{\MCD}_1 \},\label{set:D3}\\
\MCD_4 & := \left\{
\begin{array}{ll}
\displaystyle \left\{z\in\CS:
\left|\frac{8}{5}-z\right|<\frac{6}{5}\right\},
& p=2,3,4,\\
\displaystyle \left\{z\in\CS: |\arg(z)|<\frac{\pi}{6},
\left|\frac{5p-13}{4(p-3)} - z\right| <
\frac{43p-105}{48(p-3)}\right\}, & p\geq5. \nonumber
\end{array}
\right.
\end{align}
In Figure \ref{fig:NM_PraConvReg}, we present the regions $\MCR_1$
and $\MCR_1^{\text{N}}$ defined in (\ref{set:R_NM}) and
(\ref{set:R_N_pra}), respectively, for Newton's method. We can
observe that the new region $\MCR_1^{\text{N}}$ is acceptable
approximation to $\MCR_1$. So instead of using $\MCR_1$, in Section
4, we will use $\MCR_1^{\text{N}}$ in our algorithm and numerical
experiments.
\end{remark}


\begin{figure}[h!]
\centering
\subfigure[]{\includegraphics[width=0.3\textwidth]{fig_NM_PraConvReg_p3m20.eps}}
\subfigure[]{\includegraphics[width=0.3\textwidth]{fig_NM_PraConvReg_p5m20.eps}}
\subfigure[]{\includegraphics[width=0.3\textwidth]{fig_NM_PraConvReg_p11m20.eps}}\\
\subfigure[]{\includegraphics[width=0.3\textwidth]{fig_NM_PraConvReg_p20m20.eps}}
\subfigure[]{\includegraphics[width=0.3\textwidth]{fig_NM_PraConvReg_p50m20.eps}}
\subfigure[]{\includegraphics[width=0.3\textwidth]{fig_NM_PraConvReg_p100m20.eps}}\\
\caption{For fixed $m=20$ and $p=2,5,11,20,50,100$, the actual
convergence regions $\MCR_1$ defined in (\ref{set:R_NM}) (the union
of the red and blue parts) and the approximate convergence regions
$\MCR_1^{\text{N}}$ defined in (\ref{set:R_N_pra}) (the yellow
parts).}\label{fig:NM_PraConvReg}
\end{figure}








Next, we give some numerical examples to illustrate that the
convergence results obtained in Theorems \ref{th:MatrixNMCon} and
\ref{th:MatrixHMCon} are better than the existing ones. For
simplicity, in the following tests, we denote

\begin{enumerate}
\item
SN: Schur-Newton algorithm by choosing $\MCR_1^{\text{N}}$ defined
in (\ref{set:R_N_pra}) as the region $\MCR$ in step 2 and the
coupled Newton iteration (\ref{it:CoupledNM}) as the iterative
method in step 4;
\item
SN-old: Schur-Newton algorithm by choosing $\MCD_3$ defined in
(\ref{set:D3}) as the region $\MCR$ in step 2 and the coupled Newton
iteration (\ref{it:CoupledNM}) as the iterative method in step 4;
\item
SH: Schur-Halley algorithm by choosing $\MCR_2^{\text{H}}$ defined
in (\ref{set:R_H_pra}) as the region $\MCR$ in step 2 and the
coupled Halley iteration (\ref{it:CoupledHM}) as the iterative
method in step 4;
\item
SH-old: Schur-Halley algorithm by choosing the disk $\{z\in\CS:
|8/5-z|\leq1\}$ as the region $\MCR$ in step 2 and the coupled
Halley iteration (\ref{it:CoupledHM}) as the iterative method in
step 4.
\end{enumerate}
Recall that the convergence regions of SN-old and SH-old are
presented by Guo in \cite{Guo2010} and Iannazzo in
\cite{Iannazzo2008}, respectively. The iterations in the above four
algorithms are stopped when $\|N_k - \I\| < \sqrt{n}u/2$, where $n$
is the size of $A$ and $u = 2^{-52} \approx2.2204\me-16$.


Our numerical experiments were carried out in MATLAB 7.0 running on
a PC Intel Pentium P6200 of 2.13 GHz CPU. To measure of the quality
of a computed solution $X$, we use the relative residual $\rho_A(X)$
and relative error $\textup{err}(X)$ as follows:
\begin{equation}
\label{eq:RelResErr} \rho_A(X) = \frac{\|A - X^p\|}{\|X\| \|\sum_{j
= 0}^{p - 1} \tran{(X^{p - 1 - j})}\otimes X^j\|}, \ \ \
\textup{err}(X) = \frac{\|A - X^p\|}{\|A\|},
\end{equation}
where $\otimes$ denotes the Kronecker product and $\|\cdot\|$
denotes the Frobenius norm. Note that the relative residual
$\rho_A(X)$ (given in \cite{GuoHigham2006}) is more practically
useful definition of relative residual (e.g., for testing purposes)
than relative error and that the averaged CPU time computed by the
standard MATLAB function \textsf{cputime}. The averaged time was
computed by repeating 100 times for each test matrix. Moreover, we
use `iter' to stand for the number of the iterations.
